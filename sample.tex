\documentclass[xetex, aspectratio=169]{beamer}
\usepackage[utf8]{inputenc}
\usepackage[T1]{fontenc}

% for long verbatim lines
\usepackage{listings}
\lstset{basicstyle=\ttfamily,breaklines=true}

%% Title and authors
\title{A Non-intrusive Beamer Theme Focusing on Content}
\subtitle{simple theme}
\date{\today}
\author{\underline{Adarsh}}

% Color palette 
\definecolor{AlertColor}{HTML}{DA4D45}
\definecolor{MyTeal}{HTML}{00796B}
\definecolor{MyBlue}{HTML}{03A9F4}
\definecolor{MyPink}{HTML}{E91E63}
\definecolor{MyPurple}{HTML}{6A1B9A}
\definecolor{MyLime}{HTML}{C6FF00}
\definecolor{MyGreen}{HTML}{388E3C}

% two styles light and dark
% light is default
\usetheme[style=dark]{simple}

\begin{document}

\begin{frame}
\maketitle
\end{frame}

%% Toc in first frame - takes sections and not frames
%\begin{frame}{Table of contents}
%  \setbeamertemplate{section in toc}[sections numbered]
%  \tableofcontents[hideallsubsections]
%\end{frame}

%% Just for reference - it doesn't create a new slide
\section{Why a new theme?}

\begin{frame}[fragile]{Why are we building a new theme?}
	% examples taken from Indyk's slides
	\centering 
	\alert{``We want to have a minimal presentation where only content matters.''}
	\vspace{\baselineskip}
	\begin{columns}
		\column{0.33\textwidth}
		{
			\circleimage[width=1cm]{MyGreen}{2cm}{img/smallnotebook}
			\begin{center}
				{\Large Short}\\
				Write concise sentences.
			\end{center}
			
		}
		
		\column{0.33\textwidth}
		{
			\circleimage[width=1cm]{MyLime}{2cm}{img/simplemaze}
			\begin{center}
				{\Large Simple}\\
				Do not complicate things. %\vspace*{\baselineskip}
			\end{center}
			
		}
		
		\column{0.33\textwidth}
		{
			\circleimage[width=1cm]{MyBlue}{2cm}{img/smartidea}
			\begin{center}
				{\Large Smart}\\
				Present ideas in a smart way. %\vspace*{\baselineskip}
			\end{center}
			
		}
		
	\end{columns}	
	
\end{frame}

\section{New options}

\begin{frame}[fragile]{How is this different than others?}
\centering
	\alert{Content + Cool Tools = Awesome Presentation}\vspace*{\baselineskip}
	
		Did you see those cute circles with images?

\verb|\circleimage[width=2cm]{MyPink}{3cm}{img/awesome}| \vspace*{\baselineskip}


		\circleimage[width=2cm]{MyPink}{3cm}{img/awesome}
		
		You can get cool clip-arts from \url{https://openclipart.org}.

	
\end{frame}


\begin{frame}[fragile]{Is there more?}
	
\begin{columns}
	\column{0.5\textwidth}
	\centering
	We can include images with hint of shadows (looks better in light theme).
\verb|\shadowimage[scale=0.5]{img/visual}|
and of course \verb|\includegraphics| works too. 

	\column{0.5\textwidth}
	\centering
	\begin{figure}
		\shadowimage[scale=0.45]{img/visual}
	\end{figure}
\end{columns}

\end{frame}

\begin{frame}[fragile]{What else?}
	
	\centering
	We can use cards to present our ideas. 
	\verb|\card{MyGreen}{MyBlue}{6cm}{img/strong}{<text>}|
	\vspace{\baselineskip}
	
	\begin{columns}
		\column{0.5\textwidth}
		\centering

	\card{MyGreen}{MyBlue}{6cm}{img/strong}{
		{\Large Strength} \\
		We can explain strengths here. There may be many points which we can write here.
	}

	\vspace*{0.5\baselineskip}

	\card{MyBlue}{AlertColor}{6cm}{img/weak}{
		{\Large Weakness} \\
		We can explain weakness-es here.
	}
	
		\column{0.5\textwidth}
		\centering


		\card{MyPurple}{MyLime}{6cm}{img/opportunity}{
		{\Large Opportunities} \\
		We can explain opportunities here.
		\vspace*{2\baselineskip}
	}
	
	\vspace*{0.5\baselineskip}
	
		\card{AlertColor}{MyGreen}{6cm}{img/threat}{
		{\Large Threats} \\
		We can explain threats here.
	}
	
	\end{columns}
	
\end{frame}


\begin{frame}[fragile]{Different cards for different purposes}
	
	
	\begin{columns}
		\column{0.5\textwidth}
		\centering
		
			\card{MyBlue}{AlertColor}{6cm}{img/weak}{
			{\Large Card} \\
			A usual card with an image at corner.
		}
		\begin{lstlisting}[breaklines]
\card{MyBlue}{AlertColor}{6cm}{img/weak}{<text>}
\end{lstlisting}
	%	\vspace*{0.5\baselineskip}
		
			
	\simplecard{MyGreen}{6cm}{
		{\Large Simple Card} \\ 
		This is a simple card.
	}
	\verb|\simplecard{MyGreen}{6cm}{<text>}|
		
		\column{0.5\textwidth}
		\centering
		
		\sideimagecard{MyPurple}{6cm}{img/nature}{
			{\Large Side Image Card} \\
			This is a side image card.
		}
	
	\begin{lstlisting}[breaklines]
\sideimagecard{MyPurple}{6cm}{img/nature}{<text>}
\end{lstlisting}

		
		\imagecard{img/visual1}{6cm}{
			{\Large Image Card} \\
		This card has an image as background.
		}
		\verb|\imagecard{img/visual1}{6cm}{<text>}|
		
	\end{columns}
	
\end{frame}



\begin{frame}[fragile]{Tell me more...}
\centering 
\begin{columns}
	\column{0.5\textwidth}
	I give you a minimal theme with no clutter. If you use it wisely then only sky is the limit. 
	Unleash your {\Huge \alert{CREATIVITY}}
	\column{0.5\textwidth}
	\begin{figure}
		\includegraphics[width=4cm]{img/think}
	\end{figure}
\end{columns}

\end{frame}

\begin{frame}[fragile]{How do I use it?}
	Include this in your preamble.
\begin{verbatim}
%\usetheme[style=dark]{simple}
\usetheme[style=light]{simple} 
% or  \usetheme{simple}
\end{verbatim}

It comes in	two styles --
\begin{itemize}
	\item Dark Style
	\item \alert{Light Style (Default)}
\end{itemize}

And works like any other theme...

\begin{block}{Block}
works like a block.
\end{block}

Equations render just fine -- 
\begin{equation}
A^x + B^y = ?
\end{equation}
and so on so forth.
\end{frame}

\begin{frame}[fragile]{Can I make changes?}
Yes. You can change whatever you want. Mostly, you'd want to change the font. Go to file \verb|beamerinnerthemesimple.sty| and change fonts to the ones you like.
\begin{verbatim}
\setsansfont{Futura Book}
\setmonofont{Monaco}
\end{verbatim}
You may comment 
\begin{verbatim}
\newfontfamily\futurabold{Futura-Bold}
\end{verbatim}
\end{frame}

\begin{frame}[fragile]{So what is next?}
\centering 
\alert{I'd love to hear your thoughts, issues, questions and comments.}
\begin{columns}
	\column{0.5\textwidth}
	\begin{figure}
		\includegraphics[width=3cm]{img/questionguy}
	\end{figure}
	\column{0.5\textwidth}
		\begin{figure}
		\includegraphics[width=3cm]{img/questiongirl}
	\end{figure}
\end{columns}
\end{frame}


\end{document}