\documentclass[xetex, aspectratio=169]{beamer}
\usepackage[utf8]{inputenc}
\usepackage[T1]{fontenc}

% for long verbatim lines
\usepackage{listings}
\lstset{basicstyle=\ttfamily,breaklines=true}

%% Title and authors
\title{\futurabold A Non-intrusive Beamer Theme Focusing on Content}
\subtitle{simple theme}
\date{\today}
\author{\underline{Adarsh}}

% Color palette 
\definecolor{AlertColor}{HTML}{DA4D45}
\definecolor{MyTeal}{HTML}{00796B}
\definecolor{MyBlue}{HTML}{03A9F4}
\definecolor{MyPink}{HTML}{E91E63}
\definecolor{MyPurple}{HTML}{6A1B9A}
\definecolor{MyLime}{HTML}{CCDB39}
\definecolor{MyGreen}{HTML}{388E3C}
\definecolor{SelectColor}{HTML}{F5F5F5}
\definecolor{MyIndigo}{HTML}{3E50B4}
\definecolor{MyOrange}{HTML}{FE9700}
\definecolor{MyPeach}{HTML}{FECBBB}
\definecolor{DarkBlue}{HTML}{0287D0}
\definecolor{DarkGold}{HTML}{C69318}
% three styles light, dark and none
% light is default
\usetheme[style=none]{simple}

\begin{document}

\begin{frame}
	% \maketitle also works
\makepictitle{img/coverimage}{img/logo}{DarkGold}
\end{frame}

%% Toc in first frame - takes sections and not frames
%\begin{frame}{Table of contents}
%  \setbeamertemplate{section in toc}[sections numbered]
%  \tableofcontents[hideallsubsections]
%\end{frame}

%% Just for reference - it doesn't create a new slide
\section{Why a new theme?}
\begin{frame}
	\vspace{2\baselineskip}
	\ribbon{AlertColor}{\futurabold \Huge A New Theme}
\end{frame}

\begin{frame}[fragile]{Why are we building a new theme?}
	% examples taken from Indyk's slides
	\centering 
	\alert{``We want to have a minimal presentation where only content matters.''}
	\vspace{\baselineskip}
	\begin{columns}
		\column{0.33\textwidth}
		{
			\circleimage[width=1cm]{MyGreen}{2cm}{img/smallnotebook}
			\begin{center}
				{\Large Short}\\
				Write concise sentences.
			\end{center}
			
		}
		
		\column{0.33\textwidth}
		{
			\circleimage[width=1cm]{MyLime}{2cm}{img/simplemaze}
			\begin{center}
				{\Large Simple}\\
				Do not complicate things. %\vspace*{\baselineskip}
			\end{center}
			
		}
		
		\column{0.33\textwidth}
		{
			\circleimage[width=1cm]{MyBlue}{2cm}{img/smartidea}
			\begin{center}
				{\Large Smart}\\
				Present ideas in a smart way. %\vspace*{\baselineskip}
			\end{center}
			
		}
		
	\end{columns}	
	
\end{frame}

\begin{frame}[fragile]{What does the theme provide?}
	%\centering 
	Theme in itself is quite minimal-
	\begin{itemize}
		\item Simple background
		\item Simple titles with no blocks or navigation symbols
		\item \alert{Some Cool Tools}
	\end{itemize}
	
\end{frame}

\section{New options}
\begin{frame}
	\vspace{2\baselineskip}
	\ribbon{MyGreen}{\futurabold \Huge Tools}
\end{frame}

\begin{frame}[fragile]{Tools, you say?}
\centering
	\alert{Content + Cool Tools = Awesome Presentation}\vspace*{\baselineskip}
	
		Did you see those cute circles with images?

\verb|\circleimage[width=2cm]{MyPink}{3cm}{img/awesome}| \vspace*{\baselineskip}


		\circleimage[width=2cm]{MyPink}{3cm}{img/awesome}
		
		You can get cool clip-arts from \url{https://openclipart.org}.

	
\end{frame}

\begin{frame}[fragile]{You have already seen some..}
	\centering 
	\begin{itemize}
		\item title page: \verb|\makepictitle{img/coverimage}{img/logo}{DarkGold}|
		\item section heading: \verb|\ribbon{MyGreen}{\futurabold \Huge Tools}|
	\end{itemize}
	I tried to keep them flexible. For example, I can create a thin ribbon here by just typing \verb|\ribbon[minimum height=0]{MyBlue}{I am a ribbon.}|.
	\ribbon[minimum height=0]{MyBlue}{I am a ribbon.}

\end{frame}


\begin{frame}[fragile]{Images with shadows}
	
\begin{columns}
	\column{0.5\textwidth}
	\centering
	We can include images with hint of shadows (looks better in light theme).
\verb|\shadowimage[scale=0.5]{img/visual}|
and of course \verb|\includegraphics| works too. 

	\column{0.5\textwidth}
	\centering
	\begin{figure}
		\shadowimage[scale=0.45]{img/visual}
	\end{figure}
\end{columns}

\end{frame}

\begin{frame}[fragile]{Cards}
	
	\centering
	We can use cards to present our ideas. 
	\verb|\card{MyGreen}{MyBlue}{6cm}{img/strong}{<text>}|
	\vspace{\baselineskip}
	
	\begin{columns}
		\column{0.5\textwidth}
		\centering

	\card{MyGreen}{MyBlue}{6cm}{img/strong}{
		{\Large Strength} \\
		We can explain strengths here. There may be many points which we can write here.
	}

	\vspace*{0.5\baselineskip}

	\card{MyBlue}{AlertColor}{6cm}{img/weak}{
		{\Large Weakness} \\
		We can explain weakness-es here.
	}
	
		\column{0.5\textwidth}
		\centering


		\card{MyOrange}{MyLime}{6cm}{img/opportunity}{
		{\Large Opportunities} \\
		We can explain opportunities here.
		\vspace*{2\baselineskip}
	}
	
	\vspace*{0.5\baselineskip}
	
		\card{AlertColor}{MyGreen}{6cm}{img/threat}{
		{\Large Threats} \\
		We can explain threats here.
	}
	
	\end{columns}
	
\end{frame}


\begin{frame}[fragile]{So many cards}
	
	
	\begin{columns}
		\column{0.5\textwidth}
		\centering
		
			\card{MyBlue}{AlertColor}{6cm}{img/weak}{
			{\Large Card} \\
			A usual card with an image at corner.
		}
		\begin{lstlisting}[breaklines]
\card{MyBlue}{AlertColor}{6cm}{img/weak}{<text>}
\end{lstlisting}
	%	\vspace*{0.5\baselineskip}
		
			
	\simplecard{MyGreen}{6cm}{
		{\Large Simple Card} \\ 
		This is a simple card.
	}
	\verb|\simplecard{MyGreen}{6cm}{<text>}|
		
		\column{0.5\textwidth}
		\centering
		
		\sideimagecard{MyOrange}{6cm}{img/nature}{
			{\Large Side Image Card} \\
			This is a side image card.
		}
	
	\begin{lstlisting}[breaklines]
\sideimagecard{MyOrange}{6cm}{img/nature}{<text>}
\end{lstlisting}

		
		\imagecard{img/visual1}{6cm}{
			{\Large Image Card} \\
		This card has an image as background.
		}
		\verb|\imagecard{img/visual1}{6cm}{<text>}|
		
	\end{columns}
	
\end{frame}

\begin{frame}[fragile]{Cards can be used in many ways}
	
	\centering
	Here we use them to represent task flow of software development.
	\vspace{\baselineskip}
	
	\begin{columns}
		\column{0.25\textwidth}
		\centering
		
		\card{MyGreen}{DarkBlue}{3.5cm}{img/one}{
			{\Large PREPARE} \\
			We can explain strengths here. There may be many points which we can write here. We can explain strengths here. There may be many points which we can write here.
		}
		
		\column{0.25\textwidth}
		\centering
		
		\card{MyGreen}{DarkBlue}{3.5cm}{img/two}{
			{\Large DEVELOP} \\
			We can explain weaknesses here. There may be many points which we can write here. We can explain strengths here. There may be many points which we can write here.
		}
		
		\column{0.25\textwidth}
		\centering
		
		
		\card{MyGreen}{DarkBlue}{3.5cm}{img/three}{
			{\Large DEPLOY} \\
			We can explain opportunities here. There may be many points which we can write here. We can explain strengths here. There may be many points which we can write here.
		}
		
		\column{0.25\textwidth}
		\centering
		
		\card{MyGreen}{DarkBlue}{3.5cm}{img/four}{
			{\Large SUPPORT} \\
			We can explain threats here. There may be many points which we can write here. We can explain strengths here. There may be many points which we can write here.
		}
		
	\end{columns}
	
\end{frame}


\begin{frame}[fragile]{Fancy tables}
\begin{center}
\begin{fancytable}{Table Title}{2cm}
{
	No. & Monday   & Tuesday & Wednesday & Thursday & Friday\\
	1   & A & B & C & D & E \\
	2   & F & G & H & J & K \\
	3   & A & B & C & D & E \\
}
\end{fancytable}
\end{center}
\begin{lstlisting}
\begin{fancytable}{Table Title}{2cm}
{No. & Monday   & Tuesday & Wednesday & Thursday & Friday\\
1   & A & B & C & D & E \\
2   & F & G & H & J & K \\
3   & A & B & C & D & E \\}
\end{fancytable}
\end{lstlisting}
\end{frame}

\begin{frame}[fragile]{Fancy Tables with multi-line cells}
	\centering
	Some times some rows need more depth.
	\begin{center}
		\begin{fancytable}{Algorithm Analysis}{3cm}{
				Algorithm & Run time & Remark \\
				|[text depth=3ex]|This is a big Old algorithm & 	|[text depth=3ex]| $O(n^3)$ & 	|[text depth=3ex]|  pretty slow \\
				New algorithm & $O(n)$ & still slow \\
			}
		\end{fancytable}
	\end{center}
\begin{lstlisting}
\begin{fancytable}{Algorithm Analysis}{3cm}{
Algorithm & Run time & Remark \\
|[text depth=3ex]|This is a big Old algorithm & |[text depth=3ex]| $O(n^3)$ & |[text depth=3ex]|  pretty slow \\
New algorithm & $O(n)$ & still slow \\}
\end{fancytable}
\end{lstlisting}
\end{frame}


\begin{frame}[fragile]{Fancy Tables with selection}
	\centering
	We can select a row to highlight.
	\begin{center}
	\begin{fancytable}{Algorithm Analysis}{3cm}{
		Algorithm & Run time & Remark \\
		|[fill=SelectColor]| Our algorithm & 	|[fill=SelectColor]| $O(\log n)$ & 	|[fill=SelectColor]| fast \\
		|[text depth=3ex]|This is a big Old algorithm & 	|[text depth=3ex]| $O(n^3)$ & 	|[text depth=3ex]|  pretty slow \\
		New algorithm & $O(n)$ & still slow \\
	}
	\end{fancytable}
	\end{center}
\end{frame}

\begin{frame}[fragile]{Fancy Tables are..}
	\centering
	\alert{Even Fancier!!}
	\begin{center}
		\begin{fancytable}{Algorithm Analysis}{3cm}{
		|[]|	\checkbox{} &	Algorithm & Run time & Remark \\
		|[fill=SelectColor]|\checkumark &	|[fill=SelectColor]| Our algorithm & 	|[fill=SelectColor]| $O(\log n)$ & 	|[fill=SelectColor]| fast \\
		|[text depth=3ex]|\checkbox{} & |[text depth=3ex]|This is a big Old algorithm & 	|[text depth=3ex]| $O(n^3)$ & 	|[text depth=3ex]|  pretty slow \\
		|[]|	\checkbox{}	& New algorithm & $O(n)$ & still slow \\
			}
		\end{fancytable}
	\end{center}
\end{frame}

\begin{frame}
\vspace{2\baselineskip}
\ribbon{MyOrange}{\futurabold \Huge How Does It Work?}
\end{frame}


\begin{frame}
\centering 
\begin{columns}
	\column{0.5\textwidth}
	I give you a minimal theme with no clutter. If you use it wisely then only sky is the limit. 
	Unleash your {\Huge \alert{CREATIVITY}}.
	\column{0.5\textwidth}
	\begin{figure}
		\includegraphics[width=4cm]{img/think}
	\end{figure}
\end{columns}

\end{frame}

\begin{frame}[fragile]{How do I use it?}
	Include this in your preamble.
\begin{verbatim}
\usetheme[style=light]{simple} 
\end{verbatim}

It comes in	two styles --
\begin{itemize}
	\item Dark Style - \verb|\usetheme[style=dark]{simple} |
	\item \alert{Light Style (Default)}  - \verb|\usetheme[style=light]{simple} |
	\item No style - \verb|\usetheme[style=none]{simple} |
\end{itemize}

And works like any other theme...

\begin{block}{Block}
works like a block.
\end{block}

Equations render just fine -- 
\begin{equation}
A^x + B^y = ?
\end{equation}
and so on so forth.
\end{frame}

\begin{frame}[fragile]{Can I make changes?}
Yes. You can change whatever you want. Mostly, you'd want to change the font. Go to file \verb|beamerinnerthemesimple.sty| and change fonts to the ones you like.
\begin{verbatim}
\setsansfont{Futura Book}
\setmonofont{Monaco}
\end{verbatim}
Define a bold and italic font:
\begin{verbatim}
\newfontfamily\futurabold{Futura-Bold}
\newfontfamily\futuraitalic{Futura Std Book Oblique}
\end{verbatim}
\end{frame}

\begin{frame}
	\vspace{2\baselineskip}
	\ribbon{DarkGold}{\futurabold \Huge What Next?}
\end{frame}

\begin{frame}
\centering 
\alert{I'd love to hear your thoughts, issues, questions and comments.}
\begin{columns}
	\column{0.5\textwidth}
	\begin{figure}
		\includegraphics[width=3cm]{img/questionguy}
	\end{figure}
	\column{0.5\textwidth}
		\begin{figure}
		\includegraphics[width=3cm]{img/questiongirl}
	\end{figure}
\end{columns}
\end{frame}


\end{document}